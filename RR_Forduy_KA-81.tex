\documentclass{article}
\usepackage[utf8]{inputenc}
\usepackage[T1]{fontenc}
\usepackage[ukrainian]{babel}
\usepackage[12pt]{extsizes}
\usepackage{graphicx}
\usepackage{amsmath}
\usepackage{amsfonts}
\usepackage{multicol}
\usepackage{cmap}
\usepackage{amsthm}
\graphicspath{{pictures/}}
\DeclareGraphicsExtensions{.pdf,.png,.jpg}

\title{Теорвер}
\author{nikita.forduy }
\date{February 2020}

\usepackage{natbib}
\usepackage{graphicx}

\begin{document}
\pagestyle{empty}
\newtheorem{theorem}{Теорема}

\begin{titlepage}
    \thispagestyle{empty}
    \setlength{\parindent}{0ex} % set paragraph indenting to zero
    
    \begin{center}
      НАВЧАЛЬНО-НАУКОВИЙ КОМПЛЕКС \\
      "ІНСТИТУТ ПРИКЛАДНОГО СИСТЕМНОГО АНАЛІЗУ" \\
      НАЦІОНАЛЬНОГО ТЕХНІЧНОГО УНІВЕРСИТЕТУ УКРАЇНИ \\
      "КИЇВСЬКИЙ ПОЛІТЕХНІЧНИЙ ІНСТИТУТ ІМЕНІ ІГОРЯ СІКОРСЬКОГО" \\
      \smallskip
      КАФЕДРА МАТЕМАТИЧНИХ МЕТОДІВ СИСТЕМНОГО АНАЛІЗУ \\
    \end{center}
    \vspace{60mm}
    
    \begin{center}
      РОЗРАХУНКОВА РОБОТА \\
      з предмету "Математична статистика" \\
    \end{center}
    
    \vspace{30mm}
    

    \hfill
    \begin{minipage}{.4\linewidth}
      \begin{flushright}
        Виконав студент групи КА-81
        Фордуй Нікіта
        \smallskip
        Перевірила Каніовська І.Ю.
      \end{flushright}
    \end{minipage}
    
    \vspace{10mm}

    \vfill
    \begin{center}
      Київ 2020
    \end{center}
    
    \setlength{\parindent}{5ex} % reset paragraph indenting
\end{titlepage}

\pagestyle{plain}

\large

\end{document}
